\section{A Series of Tubes}

\subsection{Types of Tubes}

\begin{itemize}
\item open - system to user, both ends open
\item return - system to system, both ends open
\item semi-closed - system to user, system side open, user side covered (e.g. elastic material)
\item fully closed - no openings on system or user side (e.g. air bubbles, resonance chambers)
\end{itemize}

\subsection{Features of Tubes}

\begin{itemize}
\item emphasize exterior connection points
\item emphasize interior design/path of tubes (for display)
\end{itemize}

\subsection{Media in Tubes}

\begin{itemize}
\item Gas (compressible).  Air with smells.  Air with fog.
\item Liquid (incompressible).  Clear, colored, conductive, heated, flavored (smoothie!).  Fill or coat.
\item Particles.  Sparse for haptic feedback.  Dense for jamming.
\item Threadables.  EL wire, conductive wire, fiberoptic cable, muscle wire.
\end{itemize}

\subsection{Design, Fabrication, Construction}

\subsubsection{Two Tools for Design}

\textbf{designing interior paths}

\textbf{designing exterior connection points}

\subsection{Inputs}

\subsubsection{Flexing}

Much like \cite{Slyper-shape}, we can sense flexing and bending of prints made on the Objet.  We can make prints on the Objet and tunnel through them, though, without making crazytastic silicone molds.

\subsubsection{Touch}
Capacitance (digital).  

\subsubsection{Pressure}
Pressure (capacitance (ish) or fluid pressure). 

\subsubsection{Tapping}
 Tapping (audio/resonance?) carried through particular tubes, like we talked about hard tubes in a soft thing.

\subsubsection{Other stuff?}
Could be.

\subsubsection{Traditional Components}
Obviously you can hook up traditional buttons, etc., the same way as always.  With copper paint instead of traditional wiring, we can share grounds, and we don't have to solder.

\subsection{Outputs}

\subsubsection{Visual}
EL wire.  Colored liquids.  Mechanical motion by pushing light stuff with air.

\subsubsection{Aural}
Resonance chambers.  Air cavity design for sound/amplification (see passive iPhone speakers).  I mean, this is basically just 3D printing instruments, which we know has been done.

\subsubsection{Haptic}
Compressible and incompressible fluids for actuation.  Recreation of PneUIs.  Tactile output.  Adding particles to add extra feedback.

\subsubsection{Olfactory/Gustatory}
Different chambers filled with different scented/flavored fluids.  We can mix them using pipes and pressure.

\subsection{Identity}
Resonance chambers for identification.