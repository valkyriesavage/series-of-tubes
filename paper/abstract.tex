\begin{abstract}
3D printers offer extraordinary flexibilty for prototyping the shape and mechanical function of objects.  However, a 3D printer's role in prototyping interactive objects is still not clear.  We propose a general technique for supporting the rapid prototyping of interactivity by removing interior material of 3D models to form pipes and cavities. We describe this new design space of pipes and hollow chambers for interaction design, where variables include openings, topologies, and inserted media.  We then present a  tool for routing such pipes through the interior of 3D models, integrated  within a 3D modeling program.  We use two distinct routing algorithms.  The first allows users to define the pipes' terminals, and then uses A* path routing and physics-based simulation to route the paths with minimized  bending energy.  The second allows users to supply a desired internal shape which we fit a pipe route to: for this we developed a novel graph-based routing algorithm.  We present several totally tubular prototypes we created using our tool to show its flexibility and potential.
\end{abstract}