\begin{abstract}
3D printers offer extraordinary flexibilty in prototyping the shape and mechanical function of objects.  However, a 3D printer's role in prototyping interactive objects is still not clear.  We propose a general technique for adding interactivity by removing interior material to form pipes and cavities. We describe the design space of pipes and hollow chambers for interaction design, touching on of openings, topologies, and inserted media.  We present a technique and design tool for routing pipes through the interior of 3D models.  We integrated our prototype with a 3D modeling program.  We use two distinct routing algorithms.  One, for creating particular points of interaction or integrating with existing electronics on the model's surface, allows users to select pipes' terminals, then uses A* path routing and physics-based simulation to minimize the bending energy of routed paths.  The second, for creating, e.g., novel neon signs, has users supply a desired internal shape which we fit a pipe route to: for this we developed a novel graph-based routing algorithm.  We present several totally tubular prototypes we created using our tool to show its flexibility and potential.
\end{abstract}