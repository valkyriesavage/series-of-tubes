\begin{abstract}
\tovi{can be condensed}
3D printers offer extraordinary flexibilty in prototyping the shape and mechanical function of objects.  However, a 3D printer's role in prototyping interactive objects is still not clear.  We propose a general technique for adding interactivity by removing interior material to form pipes and cavities. We describe the design space of pipes and hollow chambers for interaction design: a variety of openings, topologies, and inserted media for pipes create diverse inputs and outputs.  We present a technique and design tool for routing pipes of various topologies through the interior of 3D printed parts.  Our design tool is integrated into a 3D model manipulation program.  We use two distinct routing algorithms.  One, for creating particular points of interaction or integrating with existing electronics on the model's surface, allows users to select begin and end points for pipes, then uses A* path routing and physics-based simulation to minimize the bending energy of routed paths.  The second, for creating, e.g., novel neon signs, has users supply a desired internal shape which we fit a pipe route to: for this we developed a novel routing algorithm and offer proof of key traits.  We present several totally tubular prototypes we created using this tool to show its flexibility and potential, as well as to explore new points in the pipe design space.
\end{abstract}