\begin{abstract}
3D printers offer extraordinary flexibilty in prototyping the shape and mechanical function of objects.  However, in spite of recent work, a 3D printer's role in prototyping \emph{interactivity} is still not clear.  While macro-scale digital fabrication devices cannot manufacture electronics in-place, we propose a general technique for redirecting active components through the core and onto the surface of 3D printed interactive objects using a series of tubes.  We describe the design space of tubes and hollow chambers for interaction design: there are a variety of types, topologies, and inserted media for tubes that can be leveraged to create diverse inputs and outputs.  We present a technique and design tool for routing tubes of various topologies through the interior of 3D printed parts.  Our design tool is integrated into a 3D model manipulation program.  There are two distinct routing algorithms.  One allows users to select begin and end points for tubes, then uses A* path routing and physics-based simulation to minimize the bending energy of routed paths.  The second allows users to enter a description of paths to follow: for this we developed a novel neon sign routing algorithm and offer proof of several key traits.  We present several totally tubular prototypes we created using this tool to show its flexibility and potential, as well as to explore new points in the tube design space.
\end{abstract}