%!TEX root = series of tubes.tex
\begin{abstract}
3D printers offer extraordinary flexibility for prototyping the shape and mechanical function of objects.  We investigate how 3D models can be modified to facilitate the creation of \emph{interactive} objects offering dynamic input and output.
We introduce a general technique to support rapidly prototyping interactivity by removing interior material from 3D models to form internal pipes. We describe the design space of pipes for interaction design, where variables include openings, path constraints, topologies, and inserted media.  We then present \systemnamenospace, a tool for routing internal pipes through 3D models, integrated within a 3D modeling program.
We use two distinct routing algorithms.  The first has users define pipes' terminals and uses path routing and physics-based simulation to minimize pipe bending energy, allowing easy insertion of media post-print.
The second lets users supply a desired internal shape to which it fits a pipe route: for this we developed a graph-based routing algorithm.  We present prototypes created using our tool showing its flexibility and potential.
\end{abstract}
