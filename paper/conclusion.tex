\section{Conclusion}

\bjoern{We should be more upfront in the paper that the particular sensing and actuation strategies we use are taken from prior work (for example wimmer's), but that there is still novelty in how we modify the models or redirect the input and output to different parts of the model with our pipe techniques. Otherwise I can see reviewers criticizing that we only show a collection of sensing techniques that have already been published.

it's important to frame that our techniques allow us to amplify and extend the utility of existing sensing and actuation approaches}

While today's 3D printers are not yet able to fabricate active components in-place, we suggest that 3D printed interactive devices can be created with redirection of active input and output via tubes.  In addition to describing the design space of tubes and connecting related works created through both traditional processes and rapid prototyping methods, we also presented a design tool for the creation of tubes inside arbitrary 3D models.  We fabricated several example objects using this tool to explore new points in the design space.

We are heartened by our success: it opens new interactive device prototyping opportunities both for makers with consumer-grade 3D printers and research and industrial labs with higher-end machines.  We are excited to release our tool into the wild and see what people will build.