\section{Conclusion}

While today's 3D printers are not yet able to fabricate active components in-place, we suggest that 3D printed interactive devices can be created with redirection of active input and output via pipes.  This amplifies and extends the utility of existing sensing and actuation approaches.  In addition to describing the design space of pipes, we also presented a design tool for the creation of tubes inside arbitrary 3D models and discuss many techniques that combine nicely wtih pipes.  We fabricated several example objects using this tool to explore new points in the design space.

We are heartened by our success: it opens new interactive device prototyping opportunities both for makers with consumer-grade 3D printers and research and industrial labs with higher-end machines.  We are excited to release our tool into the wild and see what people build.