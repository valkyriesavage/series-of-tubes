\section{Conclusion}
While today's 3D printers are not yet able to fabricate active components in-place, we suggest that 3D printed interactive devices can be created with redirection of active input and output via tubes.  In addition to describing the design space of tubes and connecting related works created through both traditional processes and rapid prototyping methods, we also presented a design tool for the creation of tubes inside arbitrary 3D models.  We fabricated several example objects using this tool to explore new points in the design space.

We are heartened by our success: it opens new opportunities both for makers with consumer-grade 3D printers and research and industrial labs with higher-end machines.  We are excited to release our tool into the wild and see what people will build.

Ten movies streaming across that, that Internet, and what happens to your own personal Internet? I just the other day got… an Internet [that was] sent by my staff at 10 o'clock in the morning on Friday. I got it yesterday [Tuesday]. Why? Because it got tangled up with all these things going on the Internet commercially.
[…] They want to deliver vast amounts of information over the Internet. And again, the Internet is not something that you just dump something on. It's not a big truck. It's a series of tubes. And if you don't understand, those tubes can be filled and if they are filled, when you put your message in, it gets in line and it's going to be delayed by anyone that puts into that tube enormous amounts of material, enormous amounts of material.