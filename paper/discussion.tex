\section{Discussion}

\bjoern{Right now this isn't really a discussion section. Rather it's a list of things we didn't get to fully implement. This definitely suggests merging these paragraphs with the overview of other research examples in our design space and moving that section to here.}

Pipes included in objects can have additional effects beyond their designed intentions.

Pipes included in printed objects allow for their identification.  While the pipes need not be visible (especially in the case of fully-enclosed pipes), their presence, location, network structure, and length change the physical properties of a printed object.  This includes weight, acoustic resonance, and (if conductive material is present) capacitive signature.

Both identification by recall and intentional encoding are possible.  As seen in \cite{Ono-touchandactivate}, different objects have different acoustic signatures.  Additionally, two objects that are visually identical but which have fully enclosed (or other types) of pipes on their interior can be distinguished acoustically.  Thus we can recall an object's identity once its acoustic signature has been recorded.  Similarly, the Touch\'{e} system in \cite{Sato-touche} relies on distinguishing capacitive signatures of objects, both in static states and as their boundary conditions are changed by human interaction.  Weight is a simple metric of object identity, but weights are less distinctive than acoustic or capacitive signatures, as they are represented by a single number.

We have also experimented with intentional encoding.  Semi-closed pipes can function as resonance chambers.  An object's resonant frequency can be measured by attaching a speaker and microphone to it, sweeping frequencies with the speaker, and performing a Fourier transform on the resultant signal from the microphone.  Peaks in the transformed data correspond to stronger returned impulses: the resonant frequencies of the object.  Intentional encoding in the case of capacitance is significantly more difficult, and we leave it to future work.  Encoding of weight is trivial by creating fully-enclosed chambers inside an object.

This is similar to work done by Willis, et al., in \cite{Willis-infrastructs}.  While their technique requires access a terahertz imaging tool, audio resonance identity encoding requires only a microphone and a speaker.

\valkyrie{To discuss: printing media into tubes?  Design of interiors of objects more generally?  Need to cite that paper about balancing printed objects from SIGGRAPH.}