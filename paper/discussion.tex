%!TEX root = series of tubes.tex
\section{Discussion}

The promise of 3D printing is two-fold: it enables the manufacturing of complex objects at low volumes, and it allows for the manufacturing of geometries that are impossible to create using traditional techniques like injection molding.  Printing pipes to prototype interactive objects exploits both advantages. Prototypes by definition are low-volume parts, and pipes require complex interior geometries that are very difficult or impossible to produce using other processes. Our work has focused on routing pipes that are manually filled after a print has completed. However, the fundamental operations of 3D path routing and topology specification that our tool offers will remain relevant as 3D printing technology advances. Multi-material printers are already available; as they advance and support the embedding of additional materials (e.g., conductors), \systemname can enable designers to efficiently specify where different functional materials should be placed at design time.
