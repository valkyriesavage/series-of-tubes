\begin{figure}[h]
\centering
    \includegraphics[width=3.4in]{figures/placeholder/teaser.png}
\caption{A novel neon sign (a) and a touch-sensitive brain (b) designed with our pipe tool and fabricated on a consumer-grade 3D printer.  (c) and (d) show the structure of the internal pipes generated by our tool.  \bjoern{will this be the robot or the UIST sign?} \valkyrie{arguably, the robot will be nicer-looking.  I think I'm manufacturing the UIST sign on the Makerbot.}}
\label{fig:teaser}
\end{figure}

\section{Introduction}

Makers, as well as professional designers, leverage 3D printers as tools for design work.  A wide array of objects, ranging from bicycle helmets to video game controllers, are now prototyped or even manufactured using these machines.  However, most devices fabricated by 3D printers are passive.  We propose a novel technique involving the removal of material from the interior of 3D models, prior to printing, to create pipes and other cavities.  These pipes can be filled, post-print, by a variety of media that enable input, display, and tactile feedback.  This {\em subtractive} approach is complementary to {\em additive} approaches that try to print different materials, such as conductors. While our approach requires manual assembly afterwards, it opens up a large useful design space on commonly available 3D printers, allowing makers to create objects like those in Figure \ref{fig:teaser}.

These channels introduce an entire design space of opportunities for adding interactivity to these objects. For example, copper material can fill the channels, to allow for standard electronic components to be easily integrated after the printout; electroluminescent (EL) wire can be threaded through to create computer-controlled visual output; or air can be pumped through to control haptic responses.

Pipes allow makers to build with and upon existing input/output techniques, often integrating these methods into objects they were not originally designed for.   For example, to create a touch-sensitive toy as in Figure \ref{fig:teaser}, a maker can select several locations of interest on the brain.  After creating pipes to those locations and printing her object, she can fill the pipes with conductive paint.  For sensing, she need only connect a single wire to a shared out of all the tubes, and can distinguish them via Swept Frequency Capacitve Sensing \cite{Sato-touche}.  In this way, she has only a single active element (the microcontroller setup), but her otherwise-inert 3D printed object (the brain) has several sensing locations.  Other existing sensing and actuation approaches, such as FlyEye \cite{Wimmer-flyeye} and Jamming User Interfaces \cite{Follmer-jamming}, can also be enhanced by the use of pipes: makers utilizing such I/O strategies can locate input and output at arbitrary points on an print's surface.  We leverage these existing techniques for sensing and actuation while our work's novelty is in \emph{internal modeling} for 3D printing and the \emph{redirection} of I/O to arbitrary locations on a device.

Pipes can also be used for sending media on a specified route through an object; for example, neon signs are created by evacuated glass tubes formed into a specific path.  Electricity is run through the near-vacuum, creating light.  If our maker wants to build such a sign, she can substitute EL wire for the evacuated glass: she need only design a path for it to be fed through and then create pipes along that path.

Generating these pipes, however, is not trivial.  Pipes need maximum possible bend radius to ease the insertion of media post-print.  Independent pipes cannot intersect, as this could lead to electrical shorting or other problems.  Pipes should not pass too close to the surface: consumer-grade 3D printers do not guarantee air- or water-tightness of prints, so more than one printed layer may be required to prevent leakage.  Pipes for applications like neon signs must have an Euler tour through them: this allows a single EL wire to light every edge on the pipe graph.  No tools yet exist for these design tasks.  We have developed techniques for minimizing the bending energy of, enforcing non-intersection with, and avoiding the surface for pipes, as well as expanding user-specified paths and generating Euler tours.  They are embodied in our software tool enabling makers to create pipe-powered interfaces.

Our work makes the following contributions to the field of digitally fabricated interactive objects:

\begin{itemize}
\item We lay out a design space of pipes and their openings, contained media, topologies, and design focus.
\item We offer algorithms and techniques for routing pipes, as well as a design tool implementing them.
\item We showcase a set of examples, enabled by our modeling tool and exploring new points in the design space.
\end{itemize}