\begin{figure}[h]
\centering
    \includegraphics[width=3.4in]{figures/series-of-tubes.jpg}
\caption{Here's a teaser figure of some cool stuff we did. Probably it should show one of our example objects, or we can make it really big (figure*) and show them all!}
\label{fig:teaser}
\end{figure}

\section{Introduction}
Makers, as well as professional designers, leverage 3D printers as tools for design work.  A wide array of objects, ranging from bicycle helmets to jewelry to video game controllers, are now prototyped or even manufactured using these machines.  However, most devices fabricated by 3D printers are passive: accessible printers are not yet capable of creating integrated active systems. \bjoern{not sure I understand phrase after the colon. Rewrite and explain what you mean in more detail?}

Willis, et al., ``envision a future world where interactive devices can be printed rather than assembled; a world where a device with active components is created as a single object, rather than a case enclosing circuit boards and individually assembled parts" \cite{Willis-printedoptics}.  This is a vision we eagerly share: 3D printers are capable of creating arbitrary geometries not feasible to manufacture using traditional processes, and we see these capabilities being underutilized by makers and designers.  We see many opportunities to increase the interactivity of 3D prints using today's printer technology.
\tovi{Might want some examples of what possibilities this opens up}
\bjoern{I think we're doing something a little different from Karl's vision - in all of our examples, you need to hook up extra sensors, actuators, or lights to the 3D-printed object. So there's always still assembly required. We're not really printing interactive objects in a single pass, but we're offering new ways of adding interactivity to 3D printed objects.

I'd argue the principal novelty of our work is to {\em remove material} from a model to form pipes and other cavities. We demonstrate how designers/makers can then use these pipes, e.g. by filling them with various media to enable input, display and tactile feedback. This {\em subtractive} approach is complementary to {\em additive} approaches that try to print different materials such as conductors. While our approach requires manual assembly afterwards, it opens up a large useful design space on commonly available 3D printers.}

Using specially modified printers or extra machinery, it is possible to create electronics on the surface of 3D prints \cite{optomec} \cite{Sells-reprap}.  These techniques, however, require a high capital investment and technical expertise.  In addition, they lack flexibility: the printed circuits must be routed in 2D to be created on the objects' manifold exteriors, and they can only be used to create electronic circuits. \valkyrie{what?}  \tovi{I dont follow this paragraph - I think you want  to briefly summarize what has been previousl done in interactive 3d prints, and hihglight the shortcommings and remianing challenges} \bjoern{agree. Also, probably too much focus on electronics since it's only one of a series of examples we'll demonstrate.}

In addition to electronics, many other means of interactivity exist.

\valkyrie{Points to hit: there is more than electronics to worry about, like fluids and the organic haptic feedback they afford; tangible interfaces are important (do I have to say this every time??); assembly sucks}

In this paper, we propose a new technique, where interior channels, such as hollow chambers or solid \bjoern{?}, are integrated into 3D models. Such channels introduces a entire design space of opportunities for adding interactvity to 3D printed objects. For example, copper material can fill the channels, to allow for standard electronic components to be easily integrated after the printout; air can be pumped through to create tactile output; or, Electroluminescent wire can be passed through to create interactive visual output. \bjoern{this is good - I'd make this the central argument of the intro and move it up earlier.}

Our work makes the following contributions to the field of interactive digitial fabrication. First, ... design space...Second ... modeling tool and algorithms ... Third showcase a set of examples, enabled by our modeling tool. \bjoern{I'd say more about the software contribution in the abstract. Maybe that just requires expanding these notes into a full paragraph.}