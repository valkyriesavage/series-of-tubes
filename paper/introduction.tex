\begin{figure}[h]
\centering
    \includegraphics[width=3.4in]{figures/series-of-tubes.jpg}
\caption{Here's a teaser figure of some cool stuff we did. Probably it should show one of our example objects, or we can make it really big (figure*) and show them all!}
\label{fig:teaser}
\end{figure}

\section{Introduction}
Redirection leads to reusable and iterable devices.  Televisions have long been operating on this idea: before the recent switch to digital television signals, TV signals had remained unchanged since roughly the 1920s, and the only requirement to bigger and better viewing was an upgrade of the \emph{endpoint itself}.  A larger television does not require an upgrade of the underlying broadcast system.  Similarly, the Internet, famously described by US Senator Ted Stephens (R-Alaska) as a ``series of tubes'', works on the principle of routing and distributing information to endpoints; there is no need to reduplicate information in multiple locations as long as it can be retrieved and manipulated by users at their laptops, tablets, mobile phones, or swarm devices.

We propose the application of this idea, simple endpoints redirecting information to and from complex and localized systems, to user interface design.  In particular, we believe that it lends itself to design for 3D printed devices.  Since today's 3D printers cannot fabricate the electronics required to enable most devices, their place in designing interactivity has remained unclear.  Typically, when electronics are integrated with 3D printed objects it is through a manual process, where the design of the object is tightly coupled to the form factor of the electronics.  By using channels for redirection, a simple 3D printed object can be plugged in to a complex base structure to provide interactivity.  This allows more rapid iteration on the device's look and feel, while preserving its functionality.  \valkyrie{I like this idea for the introduction much better, but the way to write it is not totally clear to me yet.  suggestions welcome.}

Willis, et al., ``envision a future world where interactive devices can be printed rather than assembled; a world where a device with active components is created as a single object, rather than a case enclosing circuit boards and individually assembled parts" \cite{Willis-printedoptics}.  This is a vision we eagerly share: 3D printers are capable of creating arbitrary geometries not feasible to manufacture using traditional processes, and makers and designers are underutilizing these capabilities.

In this paper, we propose a new technique in which interior hollow chambers and pipes are integrated into 3D printed devices to redirect interactive components from a central system to arbitrary points on a printed object's surface. Such channels introduce an entire design space of opportunities for adding interactivity to these objects. For example, copper material can fill the channels, to allow for standard electronic components to be easily integrated after the printout; air can be pumped through to create tactile output; or electroluminescent wire can be threaded through to create computer-controlled visual output.

Our work makes the following contributions to the field of interactive digital fabrication:

\begin{itemize}
\item We lay out a design space of tube-mediated interactive possibilities. 
\item We offer algorithms and techniques for routing tubes, as well as a design tool implementing them.
\item We showcase a set of examples, enabled by our modeling tool and exploring new points in the design space.
\end{itemize}