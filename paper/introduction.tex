\section{Introduction}
Makers, as well as professional designers, leverage 3D printers as tools for design work.  A wide array of objects, ranging from bicycle helmets to video game controllers, are now prototyped or even manufactured using these machines.  However, most devices fabricated by 3D printers are passive: prototyping the form, but not the interactivity, of an object.  

Due to the growing popularity of 3D printing and maker communities like Thingiverse\footnote{http://thingiverse.com}, recent work has looked at the issue of rapidly prototyping interactive objects. Existing approaches include Printed Optics \cite{Willis-printedoptics}, which uses optically clear material for light redirection, and Sauron \cite{Savage-sauron}, which is based on computer vision, but they are limited to visible (reflecting light off fingers, mechanical motion of components) input and visual (screen-like) output; they do not give makers freedom to integrate haptic feedback or other inputs or outputs.

We propose a novel technique involving the removal of material from the interior of 3D models, prior to printing, to create pipes and other cavities.  These pipes can be filled, post-print, by a variety of media that enable input, display, and tactile feedback.  This {\em subtractive} approach is complementary to {\em additive} approaches that inject custom print materials, such as conductors, during the actual printing process \cite{Sells-reprap}. While our approach requires some manual assembly after a print completes, it gives makers the opportunity to cheaply and rapidly prototype a diverse range of interactive objects on 3D printers, including hobbyist machines (Figure \ref{fig:teaser}).

We describe a new design space of pipes and hollow chambers for the purpose of interaction design, where variables include openings, topologies, and inserted media. By exploring this design space new opportunities for adding interactivity to 3D printed objects can be realized. For example, copper material can fill the channels, to allow for standard electronic components to be easily integrated after the printout; electroluminescent (EL) wire can be threaded through to create computer-controlled visual output; or air can be pumped through to control haptic responses.  In comparison to traditional exterior electronics integration, our process allows makers to preserve aesthetics by hiding wires inside the object; additionally it offers more freedom in selecting precise surface locations for, e.g., air mediated tactile output.

Existing sensing and actuation approaches, such as FlyEye \cite{Wimmer-flyeye}, Swept Frequency Capacitive Sensing \cite{Sato-touche}, and Jamming User Interfaces \cite{Follmer-jamming}, can be enhanced by the use of pipes: makers utilizing such I/O strategies can locate input and output at arbitrary points on a printed 3D object's surface.  We leverage these existing techniques for sensing and actuation while our work's novelty is in \emph{internal modeling} for 3D printing and the \emph{redirection} of I/O to arbitrary locations on a device.

Generating these pipes, however, is not trivial.  For one, most 3D modelling tools do not provide extensive tools for modelling the  {\em interior} of 3D models. Furthermore, careful consideration needs to go into the design and routing of the pipes. Pipes need maximum possible bend radius to ease the insertion of media post-print.  Independent pipes cannot intersect, as this could lead to electrical shorting or other problems.  Pipes should not pass too close to the surface: consumer-grade 3D printers do not guarantee air- or water-tightness of prints, so more than one printed layer may be required to prevent leakage.  Pipes for applications like neon signs must allow a single EL wire to light every edge on the pipe graph (i.e., have an Euler tour).  No tools currently exist for these design tasks.  

Our work makes several contributions to the field of digitally fabricated interactive objects. First, we describe a design space of how pipes can be integrated into 3D models. We then offer algorithms and techniques for routing the pipes, and integrate these tools into  a 3D modelling applicaiton. The system takes into considertation the design complexities of routing interior pipes which are described above. Finally, we showcase a set of examples, enabled by our modeling tool, which allows us to explore various points in the design space. We conclude by summarizing the input and output modalities enabled by our work, and discussing limitations and areas for future exploration. 
