%!TEX root = series of tubes.tex
%\bjoern{Leave this line in - it makes compiling easier in sublime editor}

\section{Introduction}
Makers, researchers, and professional designers (jointly ``makers" in this paper) increasingly leverage 3D printers as tools for design work.  A wide array of objects, from toy figurines to video game controllers and jet engine blades, are now prototyped or even manufactured using these machines.  Most models fabricated on 3D printers today are passive: they express the form, but not the interactivity, of an object.  

\changes{Recently, human-computer interaction researchers have begun to explore how to add interaction to 3D printed objects by creating appropriate structures inside those objects. They have introduced techniques such as printed light pipes~\cite{Willis-printedoptics} and printed hollow chambers for air pressure sensing~\cite{slyper}. These projects introduce point examples of particular techniques. In this paper we present a systematic design space of using 3D printed pipes and cavities, and introduce a design tool that enables makers to add interactivity to their models through internal pipes.}

\changes{This paper introduces a novel tool that provides path planning and other high-level functionality to support the design of interior pipes and cavities within 3D models.  These pipes can be filled, post-print, with a variety of media to enable input, display, and tactile feedback.}  This {\em subtractive} approach is complementary to {\em additive} approaches that inject custom print materials, such as conductors, during the actual printing process \cite{Sells-reprap}. While our approach requires some manual assembly after a print completes, it gives makers the opportunity to cheaply and rapidly prototype a range of interactive objects on 3D printers, including popular single-material fused-deposition modeling machines (colored objects in Figure \ref{fig:teaser}).  \changes{Our software contributions can also provide useful modeling capabilities for future multi-material printers, e.g., printers that that are able to deposit conductive materials directly, without manual assembly.}

We describe a new design space of pipes for the purpose of interaction design, where variables include openings, path constraints, topologies, and inserted media. By exploring this design space, new opportunities for adding interactivity to 3D printed objects can be realized. For example, injecting conductive paint can turn pipes into conductors, which can in turn integrate electronic components into prints; electroluminescent (EL) wire can be threaded through pipes to create computer-controlled visual output; or air can be pumped through pipes to produce haptic effects.  Existing sensing and actuation approaches, such as FlyEye \cite{Wimmer-flyeye}, swept-frequency capacitive sensing \cite{Sato-touche}, and Jamming User Interfaces \cite{Follmer-jamming}, can be expanded to new contexts using pipes: makers utilizing such I/O strategies can locate input and output at arbitrary points on a printed 3D object's surface.  We leverage these existing techniques for sensing and actuation while our work's novelty is in \emph{internal modeling} for 3D printing and the \emph{redirection} of I/O to or through arbitrary locations on a device.

%Routing pipes through the inside of objects allows makers to preserve the aesthetics of their design. %additionally it offers more freedom in selecting precise surface locations for, e.g., air mediated tactile output. \bjoern{too oblique?}

Figure \ref{fig:teaser} shows a collection of interactive objects fabricated with the help of internal pipes. These objects can sense touch, detect object presence, generate haptic feedback, or show internal illumination. They could also contain other active and passive electronic components. However, manually modeling appropriate pipes for such functionality is not trivial.  For one, many 3D modelling tools do not provide extensive tools for modelling the  {\em interior} of 3D models. Furthermore, careful consideration needs to go into the design and routing of pipes when they are used to add interactivity. Pipes need maximum possible bend radius to ease the insertion of media post-print.  Independent pipes cannot intersect, as this could lead to electrical shorting or other problems.  Pipes should not pass too close to the surface: consumer-grade 3D printers do not guarantee air- or water-tightness of prints, so more than one printed layer may be required to prevent leakage.  Pipes for applications like illuminated EL wire signs must allow a single pipe to follow a specific path (e.g., for writing text). 

We contribute PipeDream, an interactive design tool for interior pipes. Our system addresses various challenges makers are likely to encounter, such as avoiding intersections and ensuring maximum bend angles. PipeDream offers 3D point-to-point routing of internal pipes based on physical simulation of flexible rods, and a graph algorithm-based path tracing technique for path-constrained pipes. We present a set of example objects, designed with the help of our tool, that demonstrate different interaction modalities. We conclude by discussing limitations and outlining areas for future exploration. 

%Our work makes several contributions to the field of digitally fabricated interactive objects. First, we describe a design space of how pipes can be integrated into 3D models. We then offer algorithms and techniques for routing the pipes, and integrate these tools into  a 3D modelling applicaiton. The system takes into considertation the design complexities of routing interior pipes which are described above. Finally, we showcase a set of examples, enabled by our modeling tool, which allows us to explore various points in the design space. We conclude by summarizing the input and output modalities enabled by our work, and discussing limitations and areas for future exploration. 
