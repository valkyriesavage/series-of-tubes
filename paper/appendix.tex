\newpage
\appendix

\section{Routing Algorithm}

We want a semi-Eulerian graph (i.e., we want a graph in which every vertex but two have even degree) so that the desired medium can be inserted, traverse every edge, and exit the graph at a different vertex.  Let $G=(V,E_0)$ s.t. $\forall e \in E_0, weight(e)=0, start \in V $ the start point $end \in V$ the end point.  Let $e_{temp} = (start, end)$, $E' = E + e_{temp}$.

We need to connect disconnected subgraphs in $G$.  Let $G_{dis} = \{G_1, G_2, ... G_n\} \in G$ s.t. $\cup{\{G_i \in G_{dis}\}} = G$ and $G_i \cap G_j  = \{ \} \forall G_i, G_j \in G_{dis}, i\neq j$ be disconnected subgraphs in G.  Create $E_{conn} = \{e = (u, v), u \in G_i, degree(u)\%2 = 1, v \in G_j, degree(v) \%2 = 1 i \neq j, weight(e) = distance(u, v)\}$ be the set of all edges connecting odd-degree points in two distinct disconnected subgraphs.  To keep only the most desirable edges in $E_{conn}$, contract each subgraph $G_i$ to a single point $u_i$ with the shortest outgoing edges to each other subgraph: $E_{conn} = \cup_{G_i \in G_{dis}} \{e=(u_i, v), v \in G_j,$ s.t. $\forall E_{conn} \ni e_k = (u, v), u \in G_i, v \in G_j, weight(e_k) \geq weight(e)\}$.  Sort $E_{conn}$ s.t. $E_{conn} = \{e_1, e_2, ... e_n\}, weight(e_1) \leq weight(e_2) \leq \ldots \leq weight(e_n)$.

Let $E_{conn-min} = \{ \}$.  Beginning with $e_1$, add the first edge $e_i \in E_{conn}$ to $E_{conn-min}$.  Then let $E_{dupe} = \{ e = (u, v) \in E_{conn}$ s.t. $u$  is reachable from $v$ along edges $E' \cup E_{conn-min}$.  Let $E_{conn} = E_{conn} \setminus E_{dupe}$.  Repeat until $E_{conn} = \{ \}$.  Let $E' = E' \cup E_{conn-min}$.

At this stage, we want an Eulerian graph: i.e., we need every vertex to be of even degree.  Let $V_{odd} = \{ v \in V$ s.t. $degree(v) \% 2 = 1 \}$.  Create $E_{circuit} = V_{odd} \times V_{odd}$, s.t. $\forall e=(u,v) \in E_{circuit}, weight(e) = distance(u, v)$.  Find minimum matching $E_{circuit-min} \subseteq E_{circuit}$.  Let $E' = E \cup E_{circuit-min}$.

Now to remove our temporary edge and make the graph semi-Eulerian instead of Eulerian, let $E' = E' \setminus e_{temp}$.

\begin{theorem} \label{thm:graphness}
 $G' = (V, E')$ is a connected, semi-Eulerian graph for which $E \subseteq E'$.
\end{theorem}

\begin{proof}
$E \subseteq E'$ : we never remove edges from $E$ in our algorithm.  $\therefore E \subseteq E'$.

$G' = (V, E')$ is connected : if $G'$ is not connected, $\exists G_i = (V_i, E_i) \subset G', u \in V_i, v\in V \setminus V_i$ s.t. $u$ is not reachable from $v$.  $G_i \not\in G_{dis}$, because we create edges connecting $G_j, G_k \forall G_j, G_k \in G_{dis}$ and only remove an edge $e_{dupe} = (u, v)$ from $E_{conn}$ once we determine that $u$ is reachable from $v$ along edges $E' \cup E_{conn-min}$.  $\therefore G_i \not\in G_{dis}$.  This implies that $G_i$ was not initially disconnected from $G$, because by definition $G_{dis}$ is the set of all disconnected subgraphs of $G$.  We cannot have disconnected $G_i$ from $G$ because $E \subseteq E'$.  Thus, a contradiction.  $\therefore G' = (V, E')$ is connected.

$G' = (V, E')$ is semi-Eulerian : Each edge $e$ in our minimum weight matching connects a unique pair of vertices $v_i, v_j \in V_{odd}$ by the definition of minimum weight matching.  Each edge $e = (v_i, v_j) \in E_{circuit-min}$ adds one to the degree of $v_i$ and $v_j$, causing them to be of even degree. $|V_{odd}| \% 2 = 0$ by the handshake lemma, $\therefore$ all edges can be paired.  When we remove $e_{temp} = (start, end)$ from $E'$, we cause those vertices, which were of even degree by the above process, to be of odd degree.  Thus, all vertices except $start$ and $end$ (which are of odd degree) are of even degree.  $\therefore G'$ is semi-Eulerian.
\end{proof}

\section{Injection Measurements}
We injected the copper a distance of 1.16m (according to the spiral length calculator at http://www.giangrandi.ch/soft/spiral/spiral.shtml ) in a spiral whose cross-section was a square of area $9mm^2$.  I suspect that is about as far as we can go without using a vacuum at the end.

\section{Objet 260 Connex Digital Materials}
Softer materials (concentration of $> 65\%$ Tango-series material) do not easily accept the copper paint (it cracks when bent and is easy to wash off).  This may preclude flex sensors made of interior copper paint (conductive thread or other materials may be able to be used for some parts of this).

\section{Bend Radius Measurements}
\begin{itemize}
\item EL wire diameter 2.3mm: minimum bend radius .35in = 8.89mm
\item Water: uh.  None?
\item Muscle Wire: ?
\item Fiber Optic Cable: ?
\end{itemize}

\section{Material Properties}

\begin{table*}[t]
\begin{tabular}{ l || l | p{3cm} | l | p{4cm} }
  \hline
  Name & Material & Resistance & Drying Time & Application Notes\\ \hline
  CuPro-Cote Coating & Copper & 2$\Omega$/inch & O(1 day) & Syringe \\ \hline
  Spectra 360 Electrode Gel & Liquid/Electrolytes & 125k$\Omega$/inch & O(hours) & Does not conduct dry \\ \hline
  Wire Glue & Carbon/Graphite &  23.6k$\Omega$/inch & O(minutes) & Syringe, very runny \\ \hline
  Bare Conductive Electric Paint & Carbon/Graphite & 110$\Omega$/inch & O(days) & Syringe \\ \hline
  Homemade Conductive Paint & Carbon/Graphite & 120$\Omega$/inch & O(hours) & \parbox{4cm}{Too thick for syringe,\\ apply externally with brush} \\ \hline
  Conductive Thread & Steel & \parbox{3cm}{1.8$\Omega$/inch taut \\ 2.5$\Omega$/inch loose} & N/A & \parbox{4cm}{Difficult to feed through\\turns} \\ \hline
  Solder Paste & Lead & 2$\Omega$/inch & N/A & \parbox{4cm}{Too thick for syringe,\\ must bake to conduct} \\ \hline
\end{tabular}
\end{table*}