%!TEX root = series of tubes.tex
\section{Related Work}
Our contributions relate to other efforts aiming to add interactivity to digitally fabricated objects and to techniques for routing and model modification.  

\subsection{Fabrication of Interactive Devices}

Previous work has begun to investigate how to fabricate interactive objects using particular sensing and display technologies. Printed Optics \cite{Willis-printedoptics} uses light pipes to create integrated displays and senses input optically through components added at print time, while Slyper et al.,~\cite{Slyper-pressure} fabricate flexible robot armatures capable of sensing interaction via changes in air pressure. Our work is inspired by these projects, and contributes a general design tool for modelling objects that can have such capabilities.  

Sauron~\cite{Savage-sauron} tracks interactions with printed physical mechanisms by placing a camera inside hollow objects, where all components must be visible to the camera.  Our technique instead can work with arbitrary, solid models and only carves out pipes for interaction where needed. Midas~\cite{Savage-midas} generates fabrication files for custom capacitive touch sensors.  Their technique is limited to planar and unfoldable shapes as it routes connections on the surface of objects. \systemname can also be used for touch sensing, but routes conductors through the inside of objects.

%Fabrickation works towards rapid fabrication of functional objects by integrating construction kit building blocks \cite{Mueller-fabrickation}.  While we recognize that 3D printing requires significant time, we believe creating precise internal structures in 3D printed objects is a strong case for digital fabrication over manual or traditional methods. \george{wording}\bjoern{not quite sure how this fits in.}


\subsubsection{Fabrication with Printed Electronics}
Researchers are also investigating how to print conductors and place electronic components as part of additive manufacturing.  Sells, et al., \cite{Sells-reprap} use a heated syringe to deposit metal in printed 2D circuitboard channels on objects' surfaces.  Other experiments have been done by the RepRap project\footnote{\url{http://reprap.org}} using plastic filament blended with conductive materials, however the resistance is currently too high to be useful in circuitry.  Sarik, et al., \cite{Sarik-tracebrush} use aerosol jetting of conductive material onto printed objects.  Park, et al., \cite{Park-microchannels} and Majidi, et al., \cite{Majidi-curvature} inject liquid metals.   Our approach does not deposit such materials automatically, but creates structures inside a 3D model that make manual post-print injection of conductive ink possible.

\subsubsection{Interaction for Fabrication vs. Fabrication for Interaction}
HCI researchers are also developing novel interaction techniques for providing input to digital fabrication machines~\cite{Mueller-constructable,mixfab,Willis-interactive}, and new hybrid hand tools that combine manual processes with digital fabrication~\cite{rivers2012position,zoran2013freed}. These projects are complementary as they target different aspects of the design process, though it is conceivable that our pipe design tools be added to AR modeling interfaces like MixFab~\cite{mixfab}.

\subsection{Routing and Internal Modeling}
Part of our contribution comprises algorithms and techniques for routing tubes through 3D objects.
Circuitboard routing algorithms, such as Lee's maze router \cite{Lee-maze}, assume 2D surfaces. Multi-layer boards consist of stacked 2D layers connected by vias. We compute true 3D point-to-point routing though the interior of solid models with flexible paths. This approach has been used in one-off designs~\cite{Navarrette-gps} but we are not aware of design tools for this purpose.

%Similar to our path routing algorithm, Wong presents a graph-based approach for generating continuous line illustrations from images \cite{Wong-continuousline}. Its approach of linking edges into a continuous line based on semi-Eulerization is similar to ours.  However, we are not confined to the plane to avoid line-crossings, and we can create ``invisible'' lines using \emph{blocking out}\footnote{In neon sign making, segments of a tube that should not appear illuminated are covered with opaque paint. This is called "blocking out". \url{http://en.wikipedia.org/wiki/Neon_sign}} \george{too much detail and talk of our work}. Prior work used tours of nodes, rather than edges, and cast continuous line drawing as a traveling salesman problem \cite{Bosch-tsp}. In traditional neon sign making, ``routing'' is a conversation between a graphic designer and a glass bender on what is feasible and comfortable for the latter and what will help him avoid burning his hands and wasting time or glass \cite{strattman1997neon}.

Some prior work has also focused on modifying interior parts of 3D objects. 
%Takayama, et al., built a system described in \cite{Takayama-fruit} which allowed users to create internal color maps of the interior of objects, allowing them to see digital ``slices'' of their ``oranges.'' \bjoern{don't understand this} Our work focuses on 3D printable geometry rather than rendering.  
In Make it Stand~\cite{Prevost-makeitstand}, material is removed from the interior of objects with the goal of changing their center of gravity. Infrastructs~\cite{Willis-infrastructs} encodes information using interior cavities that can be sensed using terahertz imaging. Bickel, et al.,~\cite{Bickel-deformation} enforced deformation behavior with multi-material blending inside objects. Modeling internal chambers is one aspect of our work, however our focus is interaction rather than balance, identification, or deformation.  
%B\"{a}cher, et al., built an algorithm for automatically placing 3D fabricatable joints into meshes \cite{Bacher-articulated}. All these contributions are orthogonal to ours, as our focus is on precisely \emph{removing} material from printed objects, rather than blending materials or modifying components.
