\section{Related Work}

\tovi{make sure to have mini intros and conclusions for each subsection.  it's ok to discuss several papers at once if they are similar.  each paper needs a sentence to say what distinguishes us from it}

Our contributions relate to other efforts to add interactivity to digitally fabricated objects; and to techniques for routing and mode modification.  \bjoern{The related work section is taking shape but still has a bit of a "missing the forest for the trees" feeling.)}

\bjoern{Tactically, to communicate who would be is a qualified reviewer for this paper in the UIST community, we should make sure to have appropriate references to Baudisch, Poupyrev, Harrison in here}

\valkyrie{can you offer suggestions?  I can imagine the latex buttons paper, touche, acoustic barcodes, capacitive fingerprinting \cite{Harrison-capfingerprint}, printed optics for harrison/poupyrev, and for baudisch maybe.... TDR \cite{Wimmer-TDR}? fiberio \cite{Holz-fiberio}? fabrickation? I'm not sure on this.}

\subsection{Fabrication of Interactive Devices}

Previous work has investigated the integration of interactivity with the fabrication process.  Printed Optics (\cite{Willis-printedoptics}) used light pipes, solid clear tubes, to create integrated displays: our work is inspired by this and goes towards a design tool for these capabilities; Printed Optics is a part of our design space, which includes other uses of pipes.  Willis, et al., also did not offer a design tool or techniques for creating light pipes automatically.  Sauron investigated the use of computer vision to track interactions with physical mechanisms \cite{Savage-sauron}.  While Savage, et al., used mirrors for redirection, we believe that the required line-of-sight operation for their computer vision techniques may be possible using printed optics or inserted optical fibers.  Slyper, et al., \cite{Slyper-pressure} 3D printed flexible robot armatures capable of sensing interaction via changes in air pressure.  This technique is complementary to ours, as we allow for placing these interactive elements anywhere on an object's surface while they had to place them in locations accessible for electronics insertion.  General-purpose routing for connecting 2D capacitive sensors to terminals is discussed in Midas \cite{Savage-midas}.  This technique, based on circuitboard routing, are limited to planar and unfoldable shapes, and are intended for use on the surface of objects rather than through their cores.  Acoustic Barcodes explored integrating ridges into digitally fabricated objects, whose acoustic signatures when scraped provided an ID code \cite{Harrison-acoustic}.  Fabrickation works towards rapid fabrication of functional objects by integrating construction kit building blocks \cite{Mueller-fabrickation}.

\subsubsection{Fabricating Channels}

Other research has attempted to bring traditional electronics to 3D prints, either on the surface or via interior channels.  Sells, et al., in \cite{Sells-reprap} create channels on the surface of a 3D printed object which are filled with a special computer-controlled syringe of liquid solder to create connections.  Their focus is on creating self-replicating 3D printers, and their electronics are confined to traditional 2D routed circuitboards.  Sarik, et al., \cite{Sarik-tracebrush} \valkyrie{I never found a real paper of this.  The Microsoft website has a "print" of it with TEI boilerplate, but I couldn't find it in the digital library.  What's up with that?} and the Optomec company \cite{optomec} both created techniques for taking an already-printed object and spraying conductive material onto its surface.  In contrast, our technique can lead to more durable circuitry, since it is routed through the core of the object rather than on its surface.  Navarette, et al., in \cite{Navarrette-gps} created a functioning GPS whose circuit routing extended into true 3D (as compared to the 2.5D boards created by stacking traditional 2D circuitboards).  However, they created this one-off routing by hand and do not offer any general algorithms or techniques.  In the area of robotics, softer-than-skin interactive surfaces have been created using tubed interfaces fabricated from silicone molds.  \cite{Park-microchannels} and \cite{Majidi-curvature} use microchannels filled with liquid metal for pressure or curvature sensing: the metal in the channels changes resistance as the channels are stretched.  These techniques have not been used with fabrication, nor has a design tool to create such sensors been previously described.

\subsection{Routing and Internal Modeling}

Part of our contribution is in the algorithms and techniques used for routing tubes through 3D objects.

Circuitboard routing algorithms, such as Lee's maze router algorithm, work only on a 2D surface.  So-called ``3D circuitboards'' are typically 2.5D, in that they comprise a stack of 2D circuitboards connected by vias \valkyrie{citation needed}.

Similar to our neon-sign routing algorithm, Wong presents a graph-based approach for generating continuous line illustrations from images \cite{Wong-continuousline}. Its approach of linking edges into a continuous line based on semi-Eulerization is similar to ours.  However, we are not confined to the plane to avoid line-crossings, and we have additional freedom in creating invisible lines where none existed.  This is possible for us thanks to \emph{blocking out}\footnote{In neon sign making, segments of a tube that should not appear illuminated are covered with opaque paint. This is called "blocking out". \url{http://en.wikipedia.org/wiki/Neon_sign}}. Prior work used tours of nodes, rather than edges, and cast continuous line drawing as a traveling salesman problem~\cite{Bosch-tsp}.  Schweikardt, et al., also made a connection between electroluminescent wire and graph theory, though their contribution is orthogonal to ours: they provide a tangible construction kit with smart nodes for building physical graph examples in which edges are lit by EL wire \cite{Schweikardt-tei09}.  In traditional neon sign making, ``routing'' is a conversation between a graphic designer and a glass bender on what is feasible and comfortable for the latter and what will help him avoid burning his hands, wasting glass, and save bending time \cite{strattman1997neon}.

\valkyrie{add reference to fruit slicing \cite{Takayama-fruit}}

Prior work on modeling the interior of objects (as opposed to their exterior) has been limited.  In Make it Stand, material is removed from the interior of objects with the goal of changing their center of gravity \cite{Prevost-makeitstand}.  Bickel, et al., explored enforcing deformation behavior with material blending \cite{Bickel-deformation}, while B\"{a}cher, et al., built an algorithm for automatically placing 3D fabricatable joints into meshes \cite{Bacher-articulated}.  Sauron automatically modified the interior of existing mechanical objects to make them amenable to vision-based sensing \cite{Savage-sauron}, and Infrastructs created encoded interior cavities in 3D printed objects designed to be sensed with a terahertz imaging tool.  Our work differs in that we want to allow users to design pipes and integrate interactive objects with them.