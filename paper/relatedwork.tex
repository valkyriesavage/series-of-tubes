\section{Related Work}

While using a series of tubes can replicate the interaction techniques discussed (and more), our work's \emph{contribution} is more generally related to two existing research areas: fabrication and routing.

\subsection{Fabrication}

Previous work has investigated the integration of interactivity into the fabrication process.  \cite{Willis-printedoptics} used light pipes, solid clear tubes, to create integrated displays.  \cite{Savage-sauron} investigated the use of computer vision to track interactions with physical mechanisms.
\begin{itemize}
\item \cite{Savage-sauron} - Sauron.  computer vision of mechanical components obviates electronics installation.
\item \cite{Willis-printedoptics} - Printed Optics.  doing cool stuff with clear material (touch sensing and display)
\item \cite{Slyper-pressure} - robots that are made of squishy stuff where air pressure changes are sensed.  input components designed to react to different manipulations (pushing, squeezing, twisting, etc.)
\item \cite{Navarrette-gps} - I'm a little unclear on what they did, but they fabbed something with a "3D circuitboard" that has a bypass that goes into 3D.  they don't offer a routing algorithm or anything, though.
\item \cite{Sarik-tracebrush} - this uses an airspray machine to add conductive paths to surface of 3D prints
\item \cite{Optomec-main} - fancy machines that spray conductive film
\item \cite{Sells-reprap} - the RepRap people using a syringe of hot solder to squirt flat circuits into flat channels
\item \cite{icecats-conductivepaint} - an instructable I followed to make my first batch of conductive paint
\end{itemize}

\subsection{Routing}
\begin{itemize}
\item \cite{Savage-midas} - Midas.  routing in 2D to connect up capacitive sensors
\item \cite{Park-microchannels} - similar to injection of liquid metal, above
\item \cite{Majidi-curvature} - inject liquid metal into really thin tubes in a soft substrate, sense stretching by changing resistance
\end{itemize}

Our neon routing algorithm is similar to that described by Wong, et al., in \cite{Wong-continuousline}.  However, We are not confined to the plane to avoid line-crossings, and we also have additional freedom in creating lines where none existed.  Because we can shield neon post-print with black tape or material (thus rendering it invisible), we don't have the same need to avoid drawing new line segments, and we are free to create much shorter paths.  It's not quite a TSP

\subsection{Interaction Techniques}
\begin{itemize}
\item \cite{Sato-touche} - Touch\'{e}.  It's like Touch\'{e} with sound, but without sound.  SFCS.
\item \cite{Harrison-buttons} - fabricate latex + acrylic buttons and pressurize with air
\item \cite{Sodhi-aireal} - air vortex generation in free space.  air haptics.
\item \cite{Yao-pneui} - PneUIs, creating interfaces with pneumatics
\item \cite{Slyper-shape} - creating silicone bendy things with embedded electronics to sense flexing, stretching, etc., supported by those shapes.
\item \cite{Iwata-volflex} - a display made up of many balloons that inflate and deflate to change the shape
\item \cite{Ono-touchandactivate} - the Touch\'{e} with sound paper from last year's UIST
\item \cite{Follmer-jamming} - Jamming User Interfaces
\item \cite{Kim-inflatablemouse} - a basic mouse, but it inflates so you can store it and also use it more reasonably than a flat mouse
\item \cite{Hashimoto-squirming} - hold a speaker in your hands, and air pressure changes make it feel like you're holidng a living, squirming thing
\end{itemize}

\subsection{Misc}
\bjoern{These may not belong in related work, but I'm just collecting them here.}
\begin{itemize}
\item \cite{Wong-cgf11} - Wong presents a graph-based approach for generating continuous line illustrations from images. It's approach of linking edges into a continuous line based on Semi-Eulerization is similar to ours. Prior work used tours of nodes, rather than edges, and cast continuous line drawing as a traveling salesman problem~\cite{Bosch-tsp}.
\item \cite{Schweikardt-tei09} Schweikardt et al. also made a connection between electroluminescent wire and graph theory, though their contribution is orthogonal: they provide a tangible construction kit with smart nodes for building physical graph examples. Edges are lit up by EL wire.
\item \cite{strattman1997neon} - Strattman is the standard "text book" for neon lighting designers.
\item In neon sign making, segments of a tube that should not appear illuminated are covered with opaque paint. This is called "blocking out"\footnote{\url{http://en.wikipedia.org/wiki/Neon_sign}}.
\end{itemize}