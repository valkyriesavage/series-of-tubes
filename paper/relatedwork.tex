\section{Related Work}
Our contributions relate to other efforts to add interactivity to digitally fabricated objects; and to techniques for routing and model modification.  

\subsection{Fabrication of Interactive Devices}

Previous work has investigated the integration of interactivity with the fabrication process.  Printed Optics (\cite{Willis-printedoptics}) used light pipes, solid clear tubes, to create integrated displays and touch sensors: our work is inspired by this and goes towards a design tool for these capabilities.  Slyper, et al., in \cite{Slyper-pressure} 3D printed flexible robot armatures capable of sensing interaction via changes in air pressure.  This technique is complementary to ours, as we allow placing these interactive elements anywhere on an object's surface; they had to place them in locations accessible for electronics insertion. Acoustic Barcodes explored integrating ridges into fabricated objects, whose acoustic signatures when scraped provided an ID code \cite{Harrison-acoustic}; while pipes can be used for identification and to create aural feedback, this is not our only goal.  Sauron investigated computer vision to track interactions with physical mechanisms \cite{Savage-sauron}.  While Savage, et al., used mirrors for redirection, we believe that the required line-of-sight operation for their computer vision techniques may be possible using printed optics or inserted optical fibers.  General-purpose routing for connecting 2D capacitive sensors to terminals is discussed in Midas \cite{Savage-midas}.  This technique, based on circuitboard routing, is limited to planar and unfoldable shapes, and intended for use on the surface of objects rather than through their cores.  Fabrickation works towards rapid fabrication of functional objects by integrating construction kit building blocks \cite{Mueller-fabrickation}.  While we recognize that 3D printing requires significant time, we believe creating precise internal structures in 3D printed objects is a strong case for digital fabrication over manual or traditional methods.

\subsubsection{Fabricating Channels}

Other research has brought traditional electronics to 3D prints.  Sells, et al., \cite{Sells-reprap}, describe using a heated syringe to deposit metal in printed 2D circuitboard channels on objects' surfaces.  Sarik, et al., \cite{Sarik-tracebrush} and the Optomec company \cite{optomec} describe \emph{spraying} conductive material onto objects to create circuitry.  Injection of liquid metals for sensing in softer-than-skin electronics was explored by Park, et al., \cite{Park-microchannels} and Majidi, et al., \cite{Majidi-curvature}.  Navarette, et al., created a functioning GPS routed in true 3D (not multilayer boards' 2.5D), hand-built using a one-off routing.  Our work differs from these because we offer \emph{algorithms and techniques} for \emph{internal}, \emph{true 3D} routing.

\subsubsection{Interaction for Fabrication}

Other researchers are investigating interaction for fabrication (i.e., interaction techniques for controlling digital fabrication machines) rather than fabrication for interaction as we do.  This includes work like Constructable \cite{Mueller-constructable}, KidCAD \cite{Follmer-kidcad}, and Willis, et al,'s interactive interfaces \cite{Willis-interactive}. Their contributions are orthogonal to ours.

\subsection{Routing and Internal Modeling}

Part of our contribution is in the algorithms and techniques used for routing tubes through 3D objects.

Circuitboard routing algorithms, such as Lee's maze router algorithm \cite{Lee-maze}, work only on a 2D surface. Multi-layer boards are typically 2.5D, in that they comprise a stack of 2D circuitboards connected by vias.  Instead we offer true 3D point to point routing though the interior of solid models.

Similar to our path routing algorithm, Wong presents a graph-based approach for generating continuous line illustrations from images \cite{Wong-continuousline}. Its approach of linking edges into a continuous line based on semi-Eulerization is similar to ours.  However, we are not confined to the plane to avoid line-crossings, and we can create ``invisible'' lines.  This is possible for us thanks to \emph{blocking out}\footnote{In neon sign making, segments of a tube that should not appear illuminated are covered with opaque paint. This is called "blocking out". \url{http://en.wikipedia.org/wiki/Neon_sign}}. Prior work used tours of nodes, rather than edges, and cast continuous line drawing as a traveling salesman problem~\cite{Bosch-tsp}. In traditional neon sign making, ``routing'' is a conversation between a graphic designer and a glass bender on what is feasible and comfortable for the latter and what will help him avoid burning his hands and wasting time or glass \cite{strattman1997neon}.

There is prior work on modeling the interior of objects (as opposed to their exterior).  In Make it Stand, material is removed from the interior of objects with the goal of changing their center of gravity \cite{Prevost-makeitstand}.  Bickel, et al., explored enforcing deformation behavior with material blending \cite{Bickel-deformation}, while B\"{a}cher, et al., built an algorithm for automatically placing 3D fabricatable joints into meshes \cite{Bacher-articulated}.  Takayama, et al., built a system described in \cite{Takayama-fruit} which allowed users to create internal color maps of the interior of objects, allowing them to see digital ``slices'' of their ``oranges.''  Sauron automatically modified the interior of existing mechanical objects to make them amenable to vision-based sensing \cite{Savage-sauron}, and Infrastructs created encoded interior cavities in 3D printed objects designed to be sensed with a terahertz imaging tool.  All these contributions are orthogonal to ours, as our focus is on modeling pipes and hollow chambers for fabricating interactive objects.