\section{Example Objects}

To evaluate and highlight our tool's capabilities, we fabricated a set of five prototype objects designed using a Series of Tubes.  All prototypes were fabricated in a single piece, unless described otherwise.

\begin{figure*}[h!]
\centering
    \includegraphics[width=7in]{figures/examples.png}
\caption{A series of example objects created using our system.  A) shows a portable radio.  B) shows a presence-aware pen holder.  C) is a rabbit which ``breathes'' using wind pipes we built.  D) shows a maze game we created.  E) is a touch-sensitive brain toy.  \valkyrie{finalize order of example objects.  what order is best?}}
\label{fig:examples}
\end{figure*}

\tovi{Maybe a materials section 
before this - to talk about the 
specific materilas we used, 
such as the copper paint 
specifications, el lighting, etc.
} \valkyrie{I'm not clear on what you are asking for here}

\subsection{Touch-sensitive Toys (open, liquid, star, exterior)}

\tovi{go into a bit more detail about modelling , fabrication, and
assembly process, and apparatus details (especially the 
copper paint - unless discussed earlier in a materials 
seciton.}

We created a set of touch-sensitive toys and a companion app, reminiscent of the boat application in Acoustic Barcodes \cite{Harrison-acoustic} \bjoern{People will not  know what this is; they shouldn't have to look up that paper to understand. I rewrote:}. The objects --- a brain, rabbit and boat, in our example --- can be set on a base, which announces the type of object, and also the names of special features of an object when that feature is touched. For example, touching a strip on top of the brain model yields the announcement ``primary motor cortex''. The distinct touch points on each object are connected by an interior star topology of tubes filled with conductive copper paint, and touch sensing is performed via a single wire and SFCS (see Figure \ref{fig:toys}).  We built a smart base which can distinguish between the toys and also determine which toy is mounted: since each toy and each gesture has a distinct capacitive signature, we use a simple classifier trained to detect both toy and gesture based on profile.

The models we used for the toys' exteriors were all downloaded from the free website Thingiverse.  All components were fabricated on a Makerbot, support-free.  After printing, we injected copper paint (CuPro-Cote) into the interior pipes using a craft syringe.  This paint requires approximately 2 days to fully dry in this configuration (3mm diameter tubes, maximum tube length 10cm).  Our smart base is powered by an Arduino Uno running code by Instructables users DZL and madshobye \bjoern{what is that code? Maybe just mention open source SFCS code and footnote a URL?}.

\subsection{Breathing Bunny (semi-closed, gas, exterior)}
\bjoern{Start each section with the larger point why we built each example. `` (semi-closed, gas, exterior)'' is likely to short. ``To demonstrate how gases can be used to both provide sensation to the user through openings and to deform a model internally, we created a rabbit...''}
\bjoern{why do we have two bunnies in this and the prior application? Maybe we should substitute a different model for the bunny above.}

We created a rabbit with a pair of tubes that can simulate breathing (see Figure \ref{fig:breathe}) \bjoern{How does it simulate breathing? Be specific ``the bunny exhales and its abdominal area rises and falls with each breath'' or something like that}.  For this, we used a combination air/vacuum pump: one terminal creates a vacuum while the other creates positive pressure.  Our rabbit has two tubes, one open tube exiting at its mouth and one semi-closed tube capped with rubberlike material in its abdomen.  We connected one tube to each of our pump's terminals, and using a programmable power supply we can mimic a rabbit's breathing pattern.  This example was printed on our Objet in rubberlike material, with support flushed post-print. \valkyrie{this paragraph is really unclear}

\subsection{Custom Radio (open, threadable, exterior)}

A custom radio built using tubes allows users to tune to different stations and listen in.  This device uses a network of disconnected open tubes filled with conductive paint to connect traditional electronic components (two potentiometers and a speaker) to a microcontroller (in our case, an Arduino Pro Micro).  We hid the microcontroller and a battery-driven power supply in the base of the radio to make it portable.  For the radio functionality, we attached an Si4703 to the Arduino, which offers headphone-jack output.  We fabricated the radio itself on our Makerbot, support-free but cut in two pieces to allow the Arduino and battery pack to be inserted.

\subsection{Presence-aware Pen Holder (open, solid, exterior)}


Our presence-aware pen holder can distinguish which tool or tools a user has picked up (see Figure \ref{fig:pens}).  Our pen holder uses a modification of the FlyEye technique described by Wimmer in \cite{Wimmer-flyeye} and contains open tubes filled with fiber optic cables, one per pen chamber.  We use a single cable per chamber; our 6mm diameter fiber optic cables, in comparison to the fine tubes used in the original work , can send and receive through the same cable.  At the base of each tube is a QRE1113 line sensor digital breakout board, which has an integrated IR emitter and receiver.   When a pen is in its appointed place, the emitted infrared light is reflected off its bottom and travels back to the receiver, where it registers as bright.  This prototype was built by our Objet and support was flushed post-print. \valkyrie{this whole paragraph is unclear}

\subsection{Maze (fully enclosed, particulate, tree, interior)}


We created a maze game with a single particle trapped in a fully enclosed tube.  This maze is based on an SVG we created by hand and processed with our internal path tube tool.  To fabricate the tube, we created it in two halves that were fastened together via glue, due to limitations in our ability to remove support material otherwise. \bjoern{observant reviewers will call us out on this - that many of our techniques may not actually be workable in a single piece because you can't get the support material out. I think if you had a lye bath you could actually get more complicated models done, as long as you're patient and there's some entry and exit for the lye to reach all pipes.}  We created this prototype on our Makerbot in two halves, inserting a marble before closing.

\subsection{Animated Neon Sign (semi-closed, threadable, interior)}

\begin{figure}[h!]
\centering
    \includegraphics[width=3.4in]{figures/sign.png}
\caption{A two-state animated neon sign designed using our tool (a)(b) \valkyrie{Tovi: we could use your PDF hax to make this animate in place :)}.  (c) shows the SVG input files to create this sign, and (d) shows an isometric view of the tubes created by our tool. \valkyrie{d) is a lot less interesting if we are making planar tubes.  also, for c, should we show an overlay of the two states?  maybe in different colors?}}
\label{fig:neon}
\end{figure}

Neon art, perhaps best known for its association with Las Vegas, is traditionally made from hand-formed glass tubes containing neon gas.  The tubes light up when a current is passed through them.  For this type of art, the path of the tubes is of crucial importance, as it determines how the sign will look.  We designed a custom neon robot head which can be animated to ``talk''.  The piece features semi-open pipes: one end must remain open to connect the EL wire to its power source.  The pipes have been threaded with EL wire which is lit in sequence to create an animation.  Our sign was fabricated on the Objet in two halves, cutting through the plane of the sign itself, with clear material for the sign around the input design and opaque everywhere else.  We manually flushed support from the non-planar connecting pipes.  The EL wire was threaded post-print, and is controlled by an EL wire sequencer.