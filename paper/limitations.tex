\section{Limitations and Future Work}
Tubes, while a powerful tool for directing interaction to arbitrary locations on an object's surface, have their shortcomings.  Most obviously, their fundamental shortfall is that they cannot replace the active elements required for a functioning interactive device: they simply redirect sensing to different locations on an object's surface.  All tubes need to, at some point, connect to a sensor, microcontroller, or pump.

Some media are difficult to insert in tubes, or may not be compatible with all types of tubes.  Fluids in particular provide challenges.  For example, the conductive paint we used in our designs can take multiple days to dry in tubes of sufficient length.  Any liquid kept in tubes printed by the Objet can cause the exterior of the part to discolor due to seepage.  Hobbyist machines do not guarantee watertightness of their prints, so leaking of water or air can occur.

\valkyrie{Our design tool is not yet fully capable of realizing all possibilities in the design space of tubes.  It lacks the network manipulation to make mixing or splitting tubes, unless they are designed in an input vector graphics file.  Future work should also include assisting makers in the fabrication process itself by automatically cutting objects to reduce support material removal issues.  Additionally, as mentioned there are opportunities to improve the algorithm for neon sign routing, as ours does not guarantee a shortest-path routing. --------- this paragraph breaks the latex for some reason?}

We are also eager to test our tool with users and get feedback from makers who use it in the wild.