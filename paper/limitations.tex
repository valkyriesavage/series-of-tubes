\section{Limitations and Future Work}
Pipes, while a powerful tool for directing interaction to arbitrary locations on an object's surface, have their shortcomings.  They cannot replace the active elements required for a functioning interactive device: they simply redirect sensing to different locations on an object's surface.  All pipes need to, at some point, connect to a sensor, microcontroller, or pump.

Some media are difficult to insert in pipes, or may not be compatible with all types of pipes.  Threadable materials in particular can provide challenges, as they can get stuck while being fed into pipes.  Fluids also  For example, the conductive paint we used in our designs can take multiple days to dry in tubes of sufficient length.  Any liquid kept in tubes printed by the Objet can cause the exterior of the part to discolor due to seepage.  Hobbyist machines do not guarantee watertightness of their prints, so leaking of water or air can occur.

As discussed, support material removal can also be a challenge.  Particularly on multi-jetting printers, printing support-free is not possible for most geometries, and detaching printed support is not always trivial.

Our design tool is not yet fully capable of realizing all possibilities in the design space of pipes.  It lacks the network manipulation to make mixing or splitting pipes, unless they are designed in an input vector graphics file.  Since pipes are committed to the mesh one at a time, our tool does not currently optimize multiple pipes at a time: we simply make greedy pairwise decisions.  This suggests the need for a better general 3D routing algorithm with multiple traces.  Future work should also include assisting makers in the fabrication process itself by automatically cutting objects to reduce support material removal issues.  %Additionally, as mentioned there are opportunities to improve the algorithm for neon sign routing, as ours does not guarantee a shortest-path routing.

We are also eager to test our tool with users and get feedback from makers who use it in the wild.