\section{Limitations and Future Work}
Tubes, while a powerful tool for directing interaction to arbitrary locations on an object's surface, have their shortcomings.  Most obviously, their fundamental shortfall is that they cannot replace the active elements required for a functioning interactive device: they simply redirect sensing to different locations on an object's surface.  All tubes need to, at some point, connect to a sensor, microcontroller, or pump.

Some media are difficult to insert in tubes, or may not be compatible with all types of tubes.  Fluids in particular provide challenges.  For example, the conductive paint we used in our designs can take multiple days to dry in tubes of sufficient length.  Any liquid kept in tubes printed by the Objet can cause the exterior of the part to discolor due to seepage.  Hobbyist machines do not guarantee watertightness of their prints, so leaking of water or air can occur.

The physical fabrication process also leads to some considerations.  To print without support on a hobbyist machine, internal tubes cannot have overhangs of greater than $45^{\circ}$.  On the Objet, support-free prints cannot overhang more than $14^{\circ}$.  Removal of support from printed tubes can be troublesome, especially when their internal geometry and branching structures are complex.

Our design tool is not yet fully capable of realizing all possibilities in the design space of tubes.  It lacks X and Y \valkyrie{probably it will lack the network manipulation to make mixing or splitting tubes...?}.  Additionally, as mentioned there are opportunities to improve the algorithm for neon sign routing, as ours does not guarantee a shortest-path routing.  We hope to address this in future work.

\valkyrie{what other future work?  miniaturization doesn't seem right, it's possible to use on mobile phones already with their on-board sensors (mic), maybe talking about shape-changing tubes?  post-fabrication modification with drills?  that doesn't sound very convincing.}

\tovi{better software for 
automatically doign 
things you had to do 
manually
user studies of software, 
and test examples and 
feedback from makers in 
the wild.
More advanced/complex 
examples of design 
possibilities}